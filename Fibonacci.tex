\documentclass{article}
\usepackage[utf8]{inputenc}

\title{Fibonacci}
\author{anofiujelailah }
\date{September 2020}
\usepackage{listings}
\usepackage{xcolor}

\definecolor{codegreen}{rgb}{0,0.6,0}
\definecolor{codegray}{rgb}{0.5,0.5,0.5}
\definecolor{codepurple}{rgb}{0.58,0,0.82}
\definecolor{backcolour}{rgb}{0.95,0.95,0.92}

\lstdefinestyle{mystyle}{
    backgroundcolor=\color{backcolour},   
    commentstyle=\color{codegreen},
    keywordstyle=\color{magenta},
    numberstyle=\tiny\color{codegray},
    stringstyle=\color{codepurple},
    basicstyle=\ttfamily\footnotesize,
    breakatwhitespace=false,         
    breaklines=true,                 
    captionpos=b,                    
    keepspaces=true,                 
    numbers=left,                    
    numbersep=5pt,                  
    showspaces=false,                
    showstringspaces=false,
    showtabs=false,                  
    tabsize=2
}

\lstset{style=mystyle}
\begin{document}

\maketitle
To properly explain why it is a bad idea to use recursion method to find the fibonacci of a number, let's analyze the following C++ code.
\begin{lstlisting}[language=C++, caption=Recursion example]
#include <iostream>

using namespace std;
int fibonacciRecursion(int nthNumber) {
        //use recursion
        if (nthNumber == 0) {

            return 0;

        } else if (nthNumber == 1) {

            return 1;
        }   
     return fibonacciRecursion(nthNumber - 1) + fibonacciRecursion(nthNumber - 2);
    }
    int main(){
    int n = 25; 

    cout << (fibonacciRecursion(n));
}
\end{lstlisting}
This code uses the recursion algorithm to find the Fibonacci of a number. If the given number is equal to 0 and 1 we return both given numbers.\\\\
However, if the given number is greater than 0 and 1, we make two recursive calls where we add both calls with the nthNumber minus 1 and 2.\\\\
This will work perfectly when we pass integers' 1 to 5. However, higher numbers like 50, 100, and above will take so much longer. \\\\
The reason for this delay is the heavy usage of the stack memory in each recursive call.\\\\
A better approach to this would be iteration, which memorizes and stores each Fibonacci calculated.
\end{document}
